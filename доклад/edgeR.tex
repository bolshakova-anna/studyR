% Options for packages loaded elsewhere
\PassOptionsToPackage{unicode}{hyperref}
\PassOptionsToPackage{hyphens}{url}
%
\documentclass[
]{article}
\usepackage{amsmath,amssymb}
\usepackage{lmodern}
\usepackage{iftex}
\ifPDFTeX
  \usepackage[T1]{fontenc}
  \usepackage[utf8]{inputenc}
  \usepackage{textcomp} % provide euro and other symbols
\else % if luatex or xetex
  \usepackage{unicode-math}
  \defaultfontfeatures{Scale=MatchLowercase}
  \defaultfontfeatures[\rmfamily]{Ligatures=TeX,Scale=1}
\fi
% Use upquote if available, for straight quotes in verbatim environments
\IfFileExists{upquote.sty}{\usepackage{upquote}}{}
\IfFileExists{microtype.sty}{% use microtype if available
  \usepackage[]{microtype}
  \UseMicrotypeSet[protrusion]{basicmath} % disable protrusion for tt fonts
}{}
\makeatletter
\@ifundefined{KOMAClassName}{% if non-KOMA class
  \IfFileExists{parskip.sty}{%
    \usepackage{parskip}
  }{% else
    \setlength{\parindent}{0pt}
    \setlength{\parskip}{6pt plus 2pt minus 1pt}}
}{% if KOMA class
  \KOMAoptions{parskip=half}}
\makeatother
\usepackage{xcolor}
\usepackage[margin=1in]{geometry}
\usepackage{color}
\usepackage{fancyvrb}
\newcommand{\VerbBar}{|}
\newcommand{\VERB}{\Verb[commandchars=\\\{\}]}
\DefineVerbatimEnvironment{Highlighting}{Verbatim}{commandchars=\\\{\}}
% Add ',fontsize=\small' for more characters per line
\usepackage{framed}
\definecolor{shadecolor}{RGB}{248,248,248}
\newenvironment{Shaded}{\begin{snugshade}}{\end{snugshade}}
\newcommand{\AlertTok}[1]{\textcolor[rgb]{0.94,0.16,0.16}{#1}}
\newcommand{\AnnotationTok}[1]{\textcolor[rgb]{0.56,0.35,0.01}{\textbf{\textit{#1}}}}
\newcommand{\AttributeTok}[1]{\textcolor[rgb]{0.77,0.63,0.00}{#1}}
\newcommand{\BaseNTok}[1]{\textcolor[rgb]{0.00,0.00,0.81}{#1}}
\newcommand{\BuiltInTok}[1]{#1}
\newcommand{\CharTok}[1]{\textcolor[rgb]{0.31,0.60,0.02}{#1}}
\newcommand{\CommentTok}[1]{\textcolor[rgb]{0.56,0.35,0.01}{\textit{#1}}}
\newcommand{\CommentVarTok}[1]{\textcolor[rgb]{0.56,0.35,0.01}{\textbf{\textit{#1}}}}
\newcommand{\ConstantTok}[1]{\textcolor[rgb]{0.00,0.00,0.00}{#1}}
\newcommand{\ControlFlowTok}[1]{\textcolor[rgb]{0.13,0.29,0.53}{\textbf{#1}}}
\newcommand{\DataTypeTok}[1]{\textcolor[rgb]{0.13,0.29,0.53}{#1}}
\newcommand{\DecValTok}[1]{\textcolor[rgb]{0.00,0.00,0.81}{#1}}
\newcommand{\DocumentationTok}[1]{\textcolor[rgb]{0.56,0.35,0.01}{\textbf{\textit{#1}}}}
\newcommand{\ErrorTok}[1]{\textcolor[rgb]{0.64,0.00,0.00}{\textbf{#1}}}
\newcommand{\ExtensionTok}[1]{#1}
\newcommand{\FloatTok}[1]{\textcolor[rgb]{0.00,0.00,0.81}{#1}}
\newcommand{\FunctionTok}[1]{\textcolor[rgb]{0.00,0.00,0.00}{#1}}
\newcommand{\ImportTok}[1]{#1}
\newcommand{\InformationTok}[1]{\textcolor[rgb]{0.56,0.35,0.01}{\textbf{\textit{#1}}}}
\newcommand{\KeywordTok}[1]{\textcolor[rgb]{0.13,0.29,0.53}{\textbf{#1}}}
\newcommand{\NormalTok}[1]{#1}
\newcommand{\OperatorTok}[1]{\textcolor[rgb]{0.81,0.36,0.00}{\textbf{#1}}}
\newcommand{\OtherTok}[1]{\textcolor[rgb]{0.56,0.35,0.01}{#1}}
\newcommand{\PreprocessorTok}[1]{\textcolor[rgb]{0.56,0.35,0.01}{\textit{#1}}}
\newcommand{\RegionMarkerTok}[1]{#1}
\newcommand{\SpecialCharTok}[1]{\textcolor[rgb]{0.00,0.00,0.00}{#1}}
\newcommand{\SpecialStringTok}[1]{\textcolor[rgb]{0.31,0.60,0.02}{#1}}
\newcommand{\StringTok}[1]{\textcolor[rgb]{0.31,0.60,0.02}{#1}}
\newcommand{\VariableTok}[1]{\textcolor[rgb]{0.00,0.00,0.00}{#1}}
\newcommand{\VerbatimStringTok}[1]{\textcolor[rgb]{0.31,0.60,0.02}{#1}}
\newcommand{\WarningTok}[1]{\textcolor[rgb]{0.56,0.35,0.01}{\textbf{\textit{#1}}}}
\usepackage{graphicx}
\makeatletter
\def\maxwidth{\ifdim\Gin@nat@width>\linewidth\linewidth\else\Gin@nat@width\fi}
\def\maxheight{\ifdim\Gin@nat@height>\textheight\textheight\else\Gin@nat@height\fi}
\makeatother
% Scale images if necessary, so that they will not overflow the page
% margins by default, and it is still possible to overwrite the defaults
% using explicit options in \includegraphics[width, height, ...]{}
\setkeys{Gin}{width=\maxwidth,height=\maxheight,keepaspectratio}
% Set default figure placement to htbp
\makeatletter
\def\fps@figure{htbp}
\makeatother
\setlength{\emergencystretch}{3em} % prevent overfull lines
\providecommand{\tightlist}{%
  \setlength{\itemsep}{0pt}\setlength{\parskip}{0pt}}
\setcounter{secnumdepth}{-\maxdimen} % remove section numbering
\ifLuaTeX
  \usepackage{selnolig}  % disable illegal ligatures
\fi
\IfFileExists{bookmark.sty}{\usepackage{bookmark}}{\usepackage{hyperref}}
\IfFileExists{xurl.sty}{\usepackage{xurl}}{} % add URL line breaks if available
\urlstyle{same} % disable monospaced font for URLs
\hypersetup{
  pdftitle={EdgeR},
  hidelinks,
  pdfcreator={LaTeX via pandoc}}

\title{EdgeR}
\author{}
\date{\vspace{-2.5em}}

\begin{document}
\maketitle

Большакова Анна 5030102/90401

Название пакета расшифровывается как анализ цифровой экспрессии генов в
R.

Работает с целочисленными таблицами.

\includegraphics{https://sun9-22.userapi.com/impg/Jq8nRQHbzZ8aUKQcIvSUnpnRI96BKvqNyfjs6Q/Toa_5aaYOgQ.jpg?size=1845x992\&quality=96\&sign=c843d6d99232c705737dc4a376a539b8\&type=album}

\begin{itemize}
\item ~
  \hypertarget{ux440ux435ux430ux43bux438ux437ux443ux435ux442-ux440ux44fux434-ux441ux442ux430ux442ux438ux441ux442ux438ux447ux435ux441ux43aux438ux445-ux43cux435ux442ux43eux434ux43eux43bux43eux433ux438ux439-ux43eux441ux43dux43eux432ux430ux43dux43dux44bux445-ux43dux430-ux43eux442ux440ux438ux446ux430ux442ux435ux43bux44cux43dux44bux445-ux431ux438ux43dux43eux43cux438ux430ux43bux44cux43dux44bux445-ux440ux430ux441ux43fux440ux435ux434ux435ux43bux435ux43dux438ux44fux445-ux432ux43aux43bux44eux447ux430ux44f-ux44dux43cux43fux438ux440ux438ux447ux435ux441ux43aux443ux44e-ux431ux430ux439ux435ux441ux43eux432ux441ux43aux443ux44e-ux43eux446ux435ux43dux43aux443-ux442ux43eux447ux43dux44bux435-ux442ux435ux441ux442ux44b-ux43eux431ux43eux431ux449ux435ux43dux43dux44bux435-ux43bux438ux43dux435ux439ux43dux44bux435-ux43cux43eux434ux435ux43bux438-ux438-ux442ux435ux441ux442ux44b-ux43aux432ux430ux437ux438ux43fux440ux430ux432ux434ux43eux43fux43eux434ux43eux431ux438ux44f.}{%
  \subsection{Реализует ряд статистических методологий, основанных на
  отрицательных биномиальных распределениях, включая эмпирическую
  байесовскую оценку, точные тесты, обобщенные линейные модели и тесты
  квазиправдоподобия.}\label{ux440ux435ux430ux43bux438ux437ux443ux435ux442-ux440ux44fux434-ux441ux442ux430ux442ux438ux441ux442ux438ux447ux435ux441ux43aux438ux445-ux43cux435ux442ux43eux434ux43eux43bux43eux433ux438ux439-ux43eux441ux43dux43eux432ux430ux43dux43dux44bux445-ux43dux430-ux43eux442ux440ux438ux446ux430ux442ux435ux43bux44cux43dux44bux445-ux431ux438ux43dux43eux43cux438ux430ux43bux44cux43dux44bux445-ux440ux430ux441ux43fux440ux435ux434ux435ux43bux435ux43dux438ux44fux445-ux432ux43aux43bux44eux447ux430ux44f-ux44dux43cux43fux438ux440ux438ux447ux435ux441ux43aux443ux44e-ux431ux430ux439ux435ux441ux43eux432ux441ux43aux443ux44e-ux43eux446ux435ux43dux43aux443-ux442ux43eux447ux43dux44bux435-ux442ux435ux441ux442ux44b-ux43eux431ux43eux431ux449ux435ux43dux43dux44bux435-ux43bux438ux43dux435ux439ux43dux44bux435-ux43cux43eux434ux435ux43bux438-ux438-ux442ux435ux441ux442ux44b-ux43aux432ux430ux437ux438ux43fux440ux430ux432ux434ux43eux43fux43eux434ux43eux431ux438ux44f.}}
\item ~
  \hypertarget{ux43fux440ux438ux43cux435ux43dux44fux435ux442ux441ux44f-ux434ux43bux44f}{%
  \subsection{Применяется
  для}\label{ux43fux440ux438ux43cux435ux43dux44fux435ux442ux441ux44f-ux434ux43bux44f}}

  \begin{itemize}
  \item ~
    \hypertarget{rna-seq}{%
    \subsection{RNA-seq}\label{rna-seq}}
  \item ~
    \hypertarget{chip-seq}{%
    \subsection{ChIP-seq}\label{chip-seq}}
  \item ~
    \hypertarget{atacseq}{%
    \subsection{ATACseq}\label{atacseq}}
  \item ~
    \hypertarget{bisulfite-seq}{%
    \subsection{Bisulfite-seq}\label{bisulfite-seq}}
  \item ~
    \hypertarget{sage}{%
    \subsection{SAGE}\label{sage}}
  \item ~
    \hypertarget{cage}{%
    \subsection{CAGE}\label{cage}}
  \end{itemize}
\end{itemize}

\includegraphics{https://sun9-39.userapi.com/impg/9MMaI4apbAX0b5b85a-0ef-sxwxjf9QnFm0RUQ/fOAOE8Gdn_A.jpg?size=1197x673\&quality=96\&sign=d0161518aafd11284430e54bcbdd15c0\&type=album}

\hypertarget{pipiline.-ux43fux440ux438ux43cux435ux440}{%
\subsection{Pipiline.
Пример}\label{pipiline.-ux43fux440ux438ux43cux435ux440}}

Установка пакета:

\begin{Shaded}
\begin{Highlighting}[]
\ControlFlowTok{if}\NormalTok{ (}\SpecialCharTok{!}\FunctionTok{require}\NormalTok{(}\StringTok{"BiocManager"}\NormalTok{, }\AttributeTok{quietly =} \ConstantTok{TRUE}\NormalTok{))}
    \FunctionTok{install.packages}\NormalTok{(}\StringTok{"BiocManager"}\NormalTok{)}
\end{Highlighting}
\end{Shaded}

\begin{verbatim}
## Warning: пакет 'BiocManager' был собран под R версии 4.2.2
\end{verbatim}

\begin{Shaded}
\begin{Highlighting}[]
\NormalTok{BiocManager}\SpecialCharTok{::}\FunctionTok{install}\NormalTok{(}\StringTok{"edgeR"}\NormalTok{)}
\end{Highlighting}
\end{Shaded}

\begin{verbatim}
## Bioconductor version 3.16 (BiocManager 1.30.19), R 4.2.0 (2022-04-22 ucrt)
\end{verbatim}

\begin{verbatim}
## Warning: package(s) not installed when version(s) same as or greater than current; use
##   `force = TRUE` to re-install: 'edgeR'
\end{verbatim}

\begin{verbatim}
## Installation paths not writeable, unable to update packages
##   path: C:/Program Files/R/R-4.2.0/library
##   packages:
##     boot, cluster, foreign, MASS, Matrix, mgcv, nlme, nnet, rpart, survival
\end{verbatim}

\begin{verbatim}
## Old packages: 'bslib', 'digest', 'evaluate', 'jsonlite', 'knitr', 'markdown',
##   'rmarkdown', 'sass', 'xfun', 'yaml'
\end{verbatim}

Подключение библиотеки:

\begin{Shaded}
\begin{Highlighting}[]
\FunctionTok{library}\NormalTok{(edgeR)}
\end{Highlighting}
\end{Shaded}

\begin{verbatim}
## Warning: пакет 'edgeR' был собран под R версии 4.2.1
\end{verbatim}

\begin{verbatim}
## Загрузка требуемого пакета: limma
\end{verbatim}

\begin{verbatim}
## Warning: пакет 'limma' был собран под R версии 4.2.1
\end{verbatim}

\hypertarget{ux447ux442ux435ux43dux438ux435-ux442ux430ux431ux43bux438ux446ux44b}{%
\subsection{0. Чтение
таблицы}\label{ux447ux442ux435ux43dux438ux435-ux442ux430ux431ux43bux438ux446ux44b}}

\begin{Shaded}
\begin{Highlighting}[]
\FunctionTok{library}\NormalTok{(edgeR)}
\FunctionTok{load}\NormalTok{(}\StringTok{"Counts.RData"}\NormalTok{)}
\NormalTok{Counts }\OtherTok{\textless{}{-}}\NormalTok{ tmp}\SpecialCharTok{$}\NormalTok{counts}
\FunctionTok{colnames}\NormalTok{(Counts) }\OtherTok{\textless{}{-}} \FunctionTok{c}\NormalTok{(}\StringTok{"16N"}\NormalTok{, }\StringTok{"16T"}\NormalTok{, }\StringTok{"18N"}\NormalTok{, }\StringTok{"18T"}\NormalTok{, }\StringTok{"19N"}\NormalTok{, }\StringTok{"19T"}\NormalTok{)}
\FunctionTok{dim}\NormalTok{(Counts)}
\end{Highlighting}
\end{Shaded}

\begin{verbatim}
## [1] 25702     6
\end{verbatim}

\begin{Shaded}
\begin{Highlighting}[]
\FunctionTok{head}\NormalTok{(Counts)}
\end{Highlighting}
\end{Shaded}

\begin{verbatim}
##           16N 16T 18N 18T 19N 19T
## 653635      0   1   0   1   0   0
## 100422834   0   0   0   0   0   0
## 645520      0   0   0   0   0   0
## 79501       0   0   0   0   0   0
## 729737      1   0   2   2   2   1
## 100507658   0   0   0   0   0   0
\end{verbatim}

\hypertarget{create-dgelist}{%
\subsection{1. Create DGEList}\label{create-dgelist}}

!! Философия этого пакета заключается в том, что вся информация должна
содержаться в одной переменной. Когда мы позже будем использовать
функции из этого пакета, мы напишем что-то вроде d \textless-
AssessmentCommonDisp(d), что, как мы обычно предположили бы,
перезаписывает d.~Вместо этого эта функция пропускает через функцию все,
что уже было в d, но просто добавляет в список еще один элемент.

\begin{Shaded}
\begin{Highlighting}[]
\NormalTok{dgList }\OtherTok{\textless{}{-}} \FunctionTok{DGEList}\NormalTok{(}\AttributeTok{counts=}\NormalTok{Counts, }\AttributeTok{genes=}\FunctionTok{rownames}\NormalTok{(Counts))}

\NormalTok{dgList}
\end{Highlighting}
\end{Shaded}

\begin{verbatim}
## An object of class "DGEList"
## $counts
##           16N 16T 18N 18T 19N 19T
## 653635      0   1   0   1   0   0
## 100422834   0   0   0   0   0   0
## 645520      0   0   0   0   0   0
## 79501       0   0   0   0   0   0
## 729737      1   0   2   2   2   1
## 25697 more rows ...
## 
## $samples
##     group lib.size norm.factors
## 16N     1  7682564            1
## 16T     1  6402930            1
## 18N     1  6275906            1
## 18T     1  8775887            1
## 19N     1  5624046            1
## 19T     1  7426932            1
## 
## $genes
##               genes
## 653635       653635
## 100422834 100422834
## 645520       645520
## 79501         79501
## 729737       729737
## 25697 more rows ...
\end{verbatim}

\begin{Shaded}
\begin{Highlighting}[]
\NormalTok{dgList}\SpecialCharTok{$}\NormalTok{samples}
\end{Highlighting}
\end{Shaded}

\begin{verbatim}
##     group lib.size norm.factors
## 16N     1  7682564            1
## 16T     1  6402930            1
## 18N     1  6275906            1
## 18T     1  8775887            1
## 19N     1  5624046            1
## 19T     1  7426932            1
\end{verbatim}

\begin{Shaded}
\begin{Highlighting}[]
\FunctionTok{head}\NormalTok{(dgList}\SpecialCharTok{$}\NormalTok{counts) }
\end{Highlighting}
\end{Shaded}

\begin{verbatim}
##           16N 16T 18N 18T 19N 19T
## 653635      0   1   0   1   0   0
## 100422834   0   0   0   0   0   0
## 645520      0   0   0   0   0   0
## 79501       0   0   0   0   0   0
## 729737      1   0   2   2   2   1
## 100507658   0   0   0   0   0   0
\end{verbatim}

\begin{Shaded}
\begin{Highlighting}[]
\FunctionTok{head}\NormalTok{(dgList}\SpecialCharTok{$}\NormalTok{genes) }
\end{Highlighting}
\end{Shaded}

\begin{verbatim}
##               genes
## 653635       653635
## 100422834 100422834
## 645520       645520
## 79501         79501
## 729737       729737
## 100507658 100507658
\end{verbatim}

\hypertarget{filtering}{%
\subsection{2. Filtering}\label{filtering}}

В этом наборе данных около 26000 генов. Однако многие из них не будут
выражены или не будут представлено достаточным количеством прочтений,
чтобы внести свой вклад в анализ. Удаление этих генов означает, что в
конечном итоге мы имеем меньше тестов для выполнения, тем самым уменьшая
проблемы, связанные с многократным тестированием. Здесь мы сохраняем
только те гены которые представлены как минимум 1cpm (cpm=counts per
million).

\begin{Shaded}
\begin{Highlighting}[]
\NormalTok{countsPerMillion }\OtherTok{\textless{}{-}} \FunctionTok{cpm}\NormalTok{(dgList)}
\FunctionTok{summary}\NormalTok{(countsPerMillion)}
\end{Highlighting}
\end{Shaded}

\begin{verbatim}
##       16N                16T                18N                18T          
##  Min.   :    0.00   Min.   :    0.00   Min.   :    0.00   Min.   :    0.00  
##  1st Qu.:    0.00   1st Qu.:    0.00   1st Qu.:    0.00   1st Qu.:    0.00  
##  Median :    0.78   Median :    2.50   Median :    0.80   Median :    1.94  
##  Mean   :   38.91   Mean   :   38.91   Mean   :   38.91   Mean   :   38.91  
##  3rd Qu.:    9.11   3rd Qu.:   22.80   3rd Qu.:    9.24   3rd Qu.:   22.79  
##  Max.   :84286.70   Max.   :64373.19   Max.   :75536.19   Max.   :80210.01  
##       19N                 19T          
##  Min.   :     0.00   Min.   :    0.00  
##  1st Qu.:     0.00   1st Qu.:    0.00  
##  Median :     0.53   Median :    2.29  
##  Mean   :    38.91   Mean   :   38.91  
##  3rd Qu.:     6.58   3rd Qu.:   21.41  
##  Max.   :104818.84   Max.   :57001.33
\end{verbatim}

\hypertarget{normalization}{%
\subsection{3. Normalization}\label{normalization}}

важно нормализовать РНК-последовательность как внутри, так и между
образцами. edgeR использует trimmed mean of M-values (TMM) method.

\begin{Shaded}
\begin{Highlighting}[]
\NormalTok{countCheck }\OtherTok{\textless{}{-}}\NormalTok{ countsPerMillion }\SpecialCharTok{\textgreater{}} \DecValTok{1}
\FunctionTok{head}\NormalTok{(countCheck)}
\end{Highlighting}
\end{Shaded}

\begin{verbatim}
##             16N   16T   18N   18T   19N   19T
## 653635    FALSE FALSE FALSE FALSE FALSE FALSE
## 100422834 FALSE FALSE FALSE FALSE FALSE FALSE
## 645520    FALSE FALSE FALSE FALSE FALSE FALSE
## 79501     FALSE FALSE FALSE FALSE FALSE FALSE
## 729737    FALSE FALSE FALSE FALSE FALSE FALSE
## 100507658 FALSE FALSE FALSE FALSE FALSE FALSE
\end{verbatim}

\begin{Shaded}
\begin{Highlighting}[]
\NormalTok{keep }\OtherTok{\textless{}{-}} \FunctionTok{which}\NormalTok{(}\FunctionTok{rowSums}\NormalTok{(countCheck) }\SpecialCharTok{\textgreater{}=} \DecValTok{2}\NormalTok{)}
\NormalTok{dgList }\OtherTok{\textless{}{-}}\NormalTok{ dgList[keep,]}
\FunctionTok{summary}\NormalTok{(}\FunctionTok{cpm}\NormalTok{(dgList)) }\CommentTok{\#compare this to the original summary}
\end{Highlighting}
\end{Shaded}

\begin{verbatim}
##       16N                16T                18N                18T          
##  Min.   :    0.00   Min.   :    0.00   Min.   :    0.00   Min.   :    0.00  
##  1st Qu.:    1.69   1st Qu.:    5.47   1st Qu.:    1.75   1st Qu.:    4.56  
##  Median :    6.77   Median :   17.49   Median :    6.85   Median :   16.98  
##  Mean   :   67.43   Mean   :   67.27   Mean   :   67.43   Mean   :   67.36  
##  3rd Qu.:   22.91   3rd Qu.:   47.32   3rd Qu.:   23.90   3rd Qu.:   49.45  
##  Max.   :84286.70   Max.   :64373.19   Max.   :75536.19   Max.   :80210.01  
##       19N                 19T          
##  Min.   :     0.00   Min.   :    0.00  
##  1st Qu.:     1.24   1st Qu.:    5.12  
##  Median :     4.80   Median :   16.56  
##  Mean   :    67.44   Mean   :   67.32  
##  3rd Qu.:    18.14   3rd Qu.:   44.30  
##  Max.   :104818.84   Max.   :57001.33
\end{verbatim}

\hypertarget{data-exploration}{%
\subsection{4. Data Exploration}\label{data-exploration}}

Мы можем изучить отношения между выборками, создав график на основе
многомерного масштабирования.

\begin{Shaded}
\begin{Highlighting}[]
\FunctionTok{plotMDS}\NormalTok{(dgList)}
\end{Highlighting}
\end{Shaded}

\includegraphics{edgeR_files/figure-latex/unnamed-chunk-7-1.pdf} \#\# 5.
Setting up the Model

Теперь мы готовы настроить модель! Сначала нам нужно указать нашу
матрицу проекта, которая описывает настройку эксперимент.

\begin{Shaded}
\begin{Highlighting}[]
\NormalTok{sampleType}\OtherTok{\textless{}{-}} \FunctionTok{rep}\NormalTok{(}\StringTok{"N"}\NormalTok{, }\FunctionTok{ncol}\NormalTok{(dgList)) }\CommentTok{\#N=normal; T=tumour}
\NormalTok{sampleType[}\FunctionTok{grep}\NormalTok{(}\StringTok{"T"}\NormalTok{, }\FunctionTok{colnames}\NormalTok{(dgList))] }\OtherTok{\textless{}{-}} \StringTok{"T"}
\CommentTok{\#\textquotesingle{}grep\textquotesingle{} is a string matching function.}
\NormalTok{sampleReplicate }\OtherTok{\textless{}{-}} \FunctionTok{paste}\NormalTok{(}\StringTok{"S"}\NormalTok{, }\FunctionTok{rep}\NormalTok{(}\DecValTok{1}\SpecialCharTok{:}\DecValTok{3}\NormalTok{, }\AttributeTok{each=}\DecValTok{2}\NormalTok{), }\AttributeTok{sep=}\StringTok{""}\NormalTok{)}

\NormalTok{designMat }\OtherTok{\textless{}{-}} \FunctionTok{model.matrix}\NormalTok{(}\SpecialCharTok{\textasciitilde{}}\NormalTok{sampleReplicate }\SpecialCharTok{+}\NormalTok{ sampleType)}
\NormalTok{designMat}
\end{Highlighting}
\end{Shaded}

\begin{verbatim}
##   (Intercept) sampleReplicateS2 sampleReplicateS3 sampleTypeT
## 1           1                 0                 0           0
## 2           1                 0                 0           1
## 3           1                 1                 0           0
## 4           1                 1                 0           1
## 5           1                 0                 1           0
## 6           1                 0                 1           1
## attr(,"assign")
## [1] 0 1 1 2
## attr(,"contrasts")
## attr(,"contrasts")$sampleReplicate
## [1] "contr.treatment"
## 
## attr(,"contrasts")$sampleType
## [1] "contr.treatment"
\end{verbatim}

2.7 Estimating Dispersion

Нам нужно оценить параметр дисперсии для нашей отрицательной
биномиальной модели. Так как их всего несколько образцы, трудно точно
оценить дисперсию для каждого гена, поэтому нам нужен способ «обмена»
информацией между генами.

В edgeR мы используем эмпирический метод Байеса, чтобы «сжать» оценки
генетической дисперсии до общей дисперсии. (tagwise dispersion)

\begin{Shaded}
\begin{Highlighting}[]
\NormalTok{dgList }\OtherTok{\textless{}{-}} \FunctionTok{estimateGLMCommonDisp}\NormalTok{(dgList, }\AttributeTok{design=}\NormalTok{designMat)}
\NormalTok{dgList }\OtherTok{\textless{}{-}} \FunctionTok{estimateGLMTrendedDisp}\NormalTok{(dgList, }\AttributeTok{design=}\NormalTok{designMat)}
\NormalTok{dgList }\OtherTok{\textless{}{-}} \FunctionTok{estimateGLMTagwiseDisp}\NormalTok{(dgList, }\AttributeTok{design=}\NormalTok{designMat)}
\end{Highlighting}
\end{Shaded}

\begin{Shaded}
\begin{Highlighting}[]
\FunctionTok{plotBCV}\NormalTok{(dgList)}
\end{Highlighting}
\end{Shaded}

\includegraphics{edgeR_files/figure-latex/unnamed-chunk-10-1.pdf}
\emph{The biological coefficient of variation (BCV) is the square root
of the dispersion parameter in the negative binomial model.}

\hypertarget{differential-expression}{%
\subsection{7.Differential Expression}\label{differential-expression}}

\begin{Shaded}
\begin{Highlighting}[]
\NormalTok{fit }\OtherTok{\textless{}{-}} \FunctionTok{glmFit}\NormalTok{(dgList, designMat)}
\NormalTok{lrt }\OtherTok{\textless{}{-}} \FunctionTok{glmLRT}\NormalTok{(fit, }\AttributeTok{coef=}\DecValTok{4}\NormalTok{)}
\end{Highlighting}
\end{Shaded}

Теперь мы можем найти наши дифференциально выраженные гены. После
подгонки модели мы можем использовать функцию topTags() для изучения
результатов и установить пороги для идентификации подмножеств
дифференциально экспрессируемых генов.

\begin{Shaded}
\begin{Highlighting}[]
\NormalTok{edgeR\_result }\OtherTok{\textless{}{-}} \FunctionTok{topTags}\NormalTok{(lrt)}
\NormalTok{?topTags}
\end{Highlighting}
\end{Shaded}

\begin{verbatim}
## запускаю httpd сервер помощи... готово
\end{verbatim}

Наконец, мы можем построить график логарифмических изменений всех генов
и выделить те, которые экспрессируются по-разному.

\begin{Shaded}
\begin{Highlighting}[]
\NormalTok{?decideTests}
\NormalTok{deGenes }\OtherTok{\textless{}{-}} \FunctionTok{decideTestsDGE}\NormalTok{(lrt, }\AttributeTok{p=}\FloatTok{0.001}\NormalTok{)}
\NormalTok{deGenes }\OtherTok{\textless{}{-}} \FunctionTok{rownames}\NormalTok{(lrt)[}\FunctionTok{as.logical}\NormalTok{(deGenes)]}
\FunctionTok{plotSmear}\NormalTok{(lrt, }\AttributeTok{de.tags=}\NormalTok{deGenes)}
\FunctionTok{abline}\NormalTok{(}\AttributeTok{h=}\FunctionTok{c}\NormalTok{(}\SpecialCharTok{{-}}\DecValTok{1}\NormalTok{, }\DecValTok{1}\NormalTok{), }\AttributeTok{col=}\DecValTok{2}\NormalTok{)}
\end{Highlighting}
\end{Shaded}

\includegraphics{edgeR_files/figure-latex/unnamed-chunk-13-1.pdf}

\end{document}
